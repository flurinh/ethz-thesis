%*******************************************************
% Abstract
%*******************************************************
%\renewcommand{\abstractname}{Abstract}
\pdfbookmark[1]{Abstract}{Abstract}
\begingroup
\let\clearpage\relax
\let\clearpage\relax
\let\clearpage\relax

\chapter*{Abstract}

Type~I (microbial) and type~II (animal) opsins are two independent solutions to the same problem in nature: photoreception. Both bind retinal in structurally analogous pockets whose residue composition determines spectral properties through the same physics. LAMBDA (Light Absorption Modeling through Binding Domain Analysis), the central contribution of this thesis, exploits this convergence: it is a graph neural network that predicts absorption maxima from binding pocket composition, trained across both opsin families and the engineered lipocalin hCRBPII. By representing binding pockets as graphs aligned on retinal, LAMBDA treats spectral tuning as a property of the chromophore environment rather than the surrounding fold. Trained on 2,120 sequences, it achieves \SI{5.18}{\nano\metre} mean absolute error on type~II opsins and \SI{6.86}{\nano\metre} on type~I; applied to 47,700 sequences, it produces the Opsin Atlas, a dataset of predicted spectral properties spanning the known diversity of both families. Cross-family learning requires a shared structural vocabulary. Type~II opsins have Ballesteros--Weinstein numbering; type~I opsins lacked an equivalent. MOGRN (Microbial Opsin Generic Residue Numbering) fills this gap---a structure-based numbering system anchored to the conserved retinal-binding site, validated against 129 structures spanning the functional diversity of microbial rhodopsins. LAMBDA uses MOGRN positions for type~I and Ballesteros--Weinstein positions for type~II to construct binding pocket graphs from sequence alone. ProtOS, a Python framework that integrates protein data management with access to machine learning models, provides the infrastructure that makes this feasible at scale: it handles sequences, structures, embeddings, and annotations through a consistent processor architecture, routing 47,700 sequences through the full analysis pipeline without manual intervention. ProtOS-MCP extends this infrastructure through a natural language interface via the Model Context Protocol, making spectral prediction, annotation, and protein engineering workflows accessible to structural biologists through conversation rather than code. The rhodozyme workflow demonstrates ProtOS in application: ProtOS integrates structure prediction, backbone design, and sequence design models into a single pipeline to computationally design a light-gated enzyme---a rhodopsin scaffold carrying a transplanted catalytic triad whose activation wavelength is set by the retinal binding pocket.

\endgroup

\clearpage%

\begingroup
\let\clearpage\relax
\let\clearpage\relax
\let\clearpage\relax

\begin{otherlanguage}{ngerman}
\pdfbookmark[1]{Zusammenfassung}{Zusammenfassung}
\chapter*{Zusammenfassung}

Opsine vom Typ~I (mikrobiell) und Typ~II (tierisch) sind zwei unabh\"angige L\"osungen f\"ur dasselbe Problem in der Natur: Photorezeption. Beide binden Retinal in strukturell analogen Bindetaschen, deren Aminos\"aurezusammensetzung die spektralen Eigenschaften durch dieselbe Physik bestimmt. LAMBDA (Light Absorption Modeling through Binding Domain Analysis), der zentrale Beitrag dieser Arbeit, nutzt diese Konvergenz: Es ist ein Graph-Neuronales-Netz, das Absorptionsmaxima aus der Zusammensetzung der Bindetasche vorhersagt, trainiert \"uber beide Opsin-Familien und das synthetische Lipocalin hCRBPII. Durch die Darstellung von Bindetaschen als Graphen, die auf Retinal ausgerichtet sind, behandelt LAMBDA die spektrale Abstimmung als Eigenschaft der Chromophor-Umgebung und nicht der umgebenden Faltung. Trainiert auf 2120 Sequenzen erreicht es einen mittleren absoluten Fehler von \SI{5,18}{\nano\metre} bei Typ-II-Opsinen und \SI{6,86}{\nano\metre} bei Typ~I; angewandt auf 47\,700 Sequenzen erzeugt es den Opsin-Atlas, einen Datensatz vorhergesagter spektraler Eigenschaften \"uber die bekannte Vielfalt beider Familien. Familien\"ubergreifendes Lernen erfordert ein gemeinsames strukturelles Vokabular. Typ-II-Opsine verf\"ugen \"uber die Ballesteros--Weinstein-Nummerierung; Typ-I-Opsinen fehlte ein \"Aquivalent. MOGRN (Microbial Opsin Generic Residue Numbering) schliesst diese L\"ucke~--- ein strukturbasiertes Nummerierungssystem, verankert an der konservierten Retinal-Bindestelle, validiert gegen 129~Strukturen, die die funktionelle Vielfalt mikrobieller Rhodopsine abdecken. LAMBDA verwendet MOGRN-Positionen f\"ur Typ~I und Ballesteros--Weinstein-Positionen f\"ur Typ~II, um Bindetaschen-Graphen allein aus der Sequenz zu konstruieren. ProtOS, ein Python-Framework, das Proteindatenmanagement mit dem Zugang zu maschinellen Lernmodellen verbindet, stellt die Infrastruktur bereit, die dies im grossen Massstab erm\"oglicht: Es verarbeitet Sequenzen, Strukturen, Embeddings und Annotationen \"uber eine konsistente Prozessorarchitektur und leitet 47\,700 Sequenzen ohne manuelles Eingreifen durch die gesamte Analysepipeline. ProtOS-MCP erweitert diese Infrastruktur durch eine nat\"urlichsprachliche Schnittstelle \"uber das Model Context Protocol und macht spektrale Vorhersage, Annotation und Protein-Engineering-Workflows f\"ur Strukturbiologen \"uber Konversation statt Code zug\"anglich. Der Rhodozym-Workflow demonstriert ProtOS in der Anwendung: ProtOS integriert Strukturvorhersage, R\"uckgrat-Design und Sequenz-Design-Modelle in eine einzige Pipeline zur computergest\"utzten Entwicklung eines lichtgesteuerten Enzyms~--- eines Rhodopsin-Ger\"usts mit einer transplantierten katalytischen Triade, deren Aktivierungswellenl\"ange durch die Retinal-Bindetasche bestimmt wird.

\end{otherlanguage}

\endgroup

\clearpage%

\begingroup
\let\clearpage\relax
\let\clearpage\relax
\let\clearpage\relax

\begin{otherlanguage}{french}
\pdfbookmark[1]{R\'esum\'e}{Resume}
\chapter*{R\'esum\'e}

Les opsines de type~I (microbiennes) et de type~II (animales) sont deux solutions ind\'ependantes au m\^eme probl\`eme dans la nature~: la photor\'eception. Toutes deux lient le r\'etinal dans des poches de liaison structurellement analogues dont la composition en r\'esidus d\'etermine les propri\'et\'es spectrales par la m\^eme physique. LAMBDA (Light Absorption Modeling through Binding Domain Analysis), la contribution centrale de cette th\`ese, exploite cette convergence~: c'est un r\'eseau de neurones sur graphes qui pr\'edit les maxima d'absorption \`a partir de la composition de la poche de liaison, entra\^in\'e sur les deux familles d'opsines et le lipocaline synth\'etique hCRBPII. En repr\'esentant les poches de liaison comme des graphes align\'es sur le r\'etinal, LAMBDA traite l'accord spectral comme une propri\'et\'e de l'environnement du chromophore plut\^ot que du repliement environnant. Entra\^in\'e sur 2120 s\'equences, il atteint une erreur absolue moyenne de \SI{5,18}{\nano\metre} sur les opsines de type~II et \SI{6,86}{\nano\metre} sur le type~I~; appliqu\'e \`a 47\,700 s\'equences, il produit l'Opsin Atlas, un jeu de donn\'ees de propri\'et\'es spectrales pr\'edites couvrant la diversit\'e connue des deux familles. L'apprentissage inter-familles n\'ecessite un vocabulaire structurel commun. Les opsines de type~II disposent de la num\'erotation Ballesteros--Weinstein~; les opsines de type~I en \'etaient d\'epourvues. MOGRN (Microbial Opsin Generic Residue Numbering) comble cette lacune~--- un syst\`eme de num\'erotation structurel ancr\'e au site conserv\'e de liaison du r\'etinal, valid\'e sur 129~structures couvrant la diversit\'e fonctionnelle des rhodopsines microbiennes. LAMBDA utilise les positions MOGRN pour le type~I et les positions Ballesteros--Weinstein pour le type~II afin de construire des graphes de poche de liaison \`a partir de la seule s\'equence. ProtOS, un framework Python qui int\`egre la gestion des donn\'ees prot\'eiques avec l'acc\`es \`a des mod\`eles d'apprentissage automatique, fournit l'infrastructure rendant cela r\'ealisable \`a grande \'echelle~: il g\`ere s\'equences, structures, embeddings et annotations via une architecture de processeurs coh\'erente, acheminant 47\,700 s\'equences \`a travers le pipeline d'analyse complet sans intervention manuelle. ProtOS-MCP \'etend cette infrastructure par une interface en langage naturel via le Model Context Protocol, rendant la pr\'ediction spectrale, l'annotation et les workflows d'ing\'enierie prot\'eique accessibles aux biologistes structuraux par la conversation plut\^ot que par le code. Le workflow rhodozyme illustre ProtOS en application~: ProtOS int\`egre des mod\`eles de pr\'ediction de structure, de conception de squelette et de conception de s\'equence en un seul pipeline pour la conception computationnelle d'une enzyme photo-activ\'ee~--- un \'echafaudage rhodopsine portant une triade catalytique transplant\'ee dont la longueur d'onde d'activation est d\'etermin\'ee par la poche de liaison du r\'etinal.

\end{otherlanguage}

\endgroup

\clearpage%

\begingroup
\let\clearpage\relax
\let\clearpage\relax
\let\clearpage\relax

\begin{otherlanguage}{italian}
\pdfbookmark[1]{Riassunto}{Riassunto}
\chapter*{Riassunto}

Le opsine di tipo~I (microbiche) e di tipo~II (animali) sono due soluzioni indipendenti allo stesso problema in natura: la fotoricezione. Entrambe legano il retinale in tasche di legame strutturalmente analoghe la cui composizione in residui determina le propriet\`a spettrali attraverso la stessa fisica. LAMBDA (Light Absorption Modeling through Binding Domain Analysis), il contributo centrale di questa tesi, sfrutta questa convergenza: \`e una rete neurale su grafi che predice i massimi di assorbimento dalla composizione della tasca di legame, addestrata su entrambe le famiglie di opsine e sulla lipocalina ingegnerizzata hCRBPII. Rappresentando le tasche di legame come grafi allineati sul retinale, LAMBDA tratta l'accordatura spettrale come una propriet\`a dell'ambiente del cromoforo piuttosto che del ripiegamento circostante. Addestrato su 2120 sequenze, raggiunge un errore assoluto medio di \SI{5,18}{\nano\metre} sulle opsine di tipo~II e \SI{6,86}{\nano\metre} sul tipo~I; applicato a 47\,700 sequenze, produce l'Opsin Atlas, un dataset di propriet\`a spettrali predette che copre la diversit\`a nota di entrambe le famiglie. L'apprendimento inter-famiglia richiede un vocabolario strutturale condiviso. Le opsine di tipo~II dispongono della numerazione Ballesteros--Weinstein; le opsine di tipo~I ne erano prive. MOGRN (Microbial Opsin Generic Residue Numbering) colma questa lacuna~--- un sistema di numerazione basato sulla struttura, ancorato al sito conservato di legame del retinale, validato su 129~strutture che coprono la diversit\`a funzionale delle rodopsine microbiche. LAMBDA utilizza le posizioni MOGRN per il tipo~I e le posizioni Ballesteros--Weinstein per il tipo~II per costruire grafi della tasca di legame dalla sola sequenza. ProtOS, un framework Python che integra la gestione dei dati proteici con l'accesso a modelli di apprendimento automatico, fornisce l'infrastruttura che rende tutto ci\`o realizzabile su larga scala: gestisce sequenze, strutture, embeddings e annotazioni attraverso un'architettura di processori coerente, instradando 47\,700 sequenze attraverso l'intera pipeline di analisi senza intervento manuale. ProtOS-MCP estende questa infrastruttura con un'interfaccia in linguaggio naturale tramite il Model Context Protocol, rendendo la predizione spettrale, l'annotazione e i workflow di ingegneria proteica accessibili ai biologi strutturali attraverso la conversazione anzich\'e il codice. Il workflow rhodozyme dimostra ProtOS in applicazione: ProtOS integra modelli di predizione strutturale, progettazione dello scheletro e progettazione della sequenza in un'unica pipeline per la progettazione computazionale di un enzima fotoattivato~--- uno scaffold rodopsinico con una triade catalitica trapiantata la cui lunghezza d'onda di attivazione \`e determinata dalla tasca di legame del retinale.

\end{otherlanguage}

\endgroup

\vfill
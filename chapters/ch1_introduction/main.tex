\chapter{Introduction}
\label{ch:introduction}

\section{A Brief History of Protein Structures}
\label{sec:intro-history}

For most of history, proteins were invisible. Biochemists knew they existed, knew they catalyzed reactions and transmitted signals, but could not see them. X-ray crystallography changed this. In 1958, John Kendrew solved the structure of myoglobin\cite{kendrew1958} and, for the first time, researchers could see a folded protein. Myoglobin is a soluble protein floating freely in the cytoplasm, a chain of amino acids wrapped into a compact globule with a heme group nestled in a hydrophobic pocket where oxygen binds. The structure revealed how myoglobin worked.

Membrane proteins proved far harder. They have hydrophobic surfaces that contact lipids and hydrophilic surfaces exposed to water on either side of the membrane. Removed from the membrane, they aggregate, and aggregated proteins cannot form the ordered crystals that X-ray crystallography requires. Membrane protein structures remained out of reach for decades, yet these proteins handle many interactions between a cell and its environment.

Despite these challenges, the first view of a membrane protein came in 1975. Richard Henderson and Nigel Unwin used electron microscopy to image bacteriorhodopsin, a light-driven proton pump from \textit{Halobacterium salinarum}\cite{henderson1975}. The resolution was \SI{7}{\angstrom}, too coarse to resolve individual atoms, but the map showed seven rod-like densities spanning the membrane, demonstrating that bacteriorhodopsin adopted a seven-transmembrane-helix fold. 

This architecture would prove significant: the same seven-helix fold also defines a major class of animal signaling proteins. In 1993, Gebhard F.X.\ Schertler, Claudio Villa, and Richard Henderson published a \SI{9}{\angstrom} projection map of bovine rhodopsin, one of the light-sensitive protein of vertebrate eyes\cite{schertler1993}. The map revealed that rhodopsin, like bacteriorhodopsin, also adopted a seven-transmembrane-helix fold. Bovine rhodopsin belongs to the G~protein coupled receptor (GPCR) family, a large group of signaling membrane proteins that activate intracellular G~proteins. A single photon triggers a structural change in rhodopsin that starts a signaling cascade, thus amplifying a tiny signal into a large cellular response. Rhodopsin was the first GPCR amenable to X-ray crystallography. In 2000, the structure of bovine rhodopsin was solved by Miyano at \SI{2.8}{\angstrom} resolution\cite{palczewski2000},  revealing how the retinal chromophore binds to the transmembrane core. In 2007, the group of Brian Kobilka solved the structure of the $\beta_2$-adrenergic receptor\cite{rasmussen2007}, introducing techniques that allowed to crystallize other GPCRs. Kobilka shared the 2012 Nobel Prize in Chemistry with Robert Lefkowitz for their work.

Crystallography requires proteins to form ordered crystals, a precondition that excludes many membrane proteins. Cryo-electron microscopy\cite{kuhlbrandt2014}, another method to elucidate protein structures, has recently allowed many more structures to be solved. This technique flash-freezes proteins in thin ice, preserving their conformations. But for decades cryo-EM produced only low-resolution images, useful to observe the overall protein shape but not atomic details. A breakthrough came around 2013 with direct electron detectors, which dramatically improved signal quality, and new algorithms that could align and average millions of particle images. Proteins that resisted crystallization---such as large complexes, flexible assemblies, membrane proteins in near-native states---could now be resolved. Henderson shared the 2017 Nobel Prize in Chemistry with Jacques Dubochet and Joachim Frank for this work.

Together, these two methods have shaped recent structural biology. The Protein Data Bank\cite{berman2000}, the largest experimental database for protein structures, grew from roughly 1,000 structures in 1993 to over 200,000 by 2024.

Structure determination investigates one protein at a time. Sequencing opened a window onto the full diversity of the protein world, at a fraction of the effort. The database UniProt\cite{uniprot2023} now contains over 200 million protein sequences. For some protein families, this revealed unexpected abundance. Microbial rhodopsins are a prime example. First identified in extreme halophilic Archaea in 1971 \cite{oesterhelt1971},  they are now recognized as among the most abundant proteins on Earth\cite{ernst2014}, with thousands of sequences spanning bacteria, archaea, algae, and fungi.

Unsurprisingly, sequence diversity far exceeds structural coverage using experimental methods. Computational models had long attempted to predict protein structures from sequence, but accuracy remained limited. In 2020, DeepMind's AlphaFold2 marked an inflection point, achieving an accuracy that made structure prediction broadly useful in structural biology\cite{jumper2021}. Within two years, DeepMind predicted structures with AlphaFold for over 200 million proteins\cite{varadi2022}. Many similar models have followed. Among them, Boltz\cite{wohlwend2024}, an open-source system to predict the structure of proteins bound to ligands (small molecules, cofactors, and interaction partners).

A similarly important class of deep learning methods are protein language models. Trained on millions of protein sequences, models like ESM-2\cite{lin2023,rives2021} and Ankh\cite{elnaggar2023ankh,elnaggar2022} produce embeddings---dense numerical vectors for each amino acid in a sequence that capture evolutionary and structural context. Averaging these per-residue vectors yields a single fingerprint for the entire protein. Because proteins with diverged sequences but similar functions produce similar embeddings, these fingerprints serve as features for downstream models that predict protein properties.

Structure prediction and protein language models have transformed what is computationally accessible, but predicted structures lack context---they come without experimental conditions, functional annotations, or consistent coordinate systems, and annotating them at scale requires automation that does not yet exist. The bottleneck has shifted from generating data to processing and understanding it. For researchers studying membrane protein families like GPCRs and microbial rhodopsins, this shift has practical consequences. These families now have hundreds of structures and are targets for drug discovery\cite{santos2017}, optogenetics\cite{deisseroth2015}, and biotechnology, yet systematic studies across their members remain difficult.


\section{Opsins}
\label{sec:intro-opsins}

Opsins exist across all domains of life, where they perform functions ranging from ion transport to signal transduction. They are seven-transmembrane proteins that become light-sensitive upon binding the chromophore retinal---a vitamin~A derivative--- through a covalent Schiff base with a lysine residue. Opsins detect light when a photon is absorbed by retinal's conjugated electron system, causing it to isomerize, which triggers a conformational change in the protein. The residues lining the binding pocket determine which wavelengths retinal absorbs, so different opsins, with different pocket residues respond to different colors of light. 

Type I (microbial) and type II (animal) opsins share this basic mechanism---retinal isomerization coupled to conformational change---yet they fall into two evolutionarily distinct families that share no detectable sequence homology\cite{ernst2014}. Both have seven transmembrane helices surrounding a retinal chromophore bound via a Schiff base linkage to a conserved lysine, but they differ substantially in helix packing, loop structures, and the specific geometry of the binding pocket. Whether this shared architecture reflects deep common ancestry or convergent evolution remains unresolved\cite{ernst2014}.

Type~I opsins are found in archaea, bacteria, algae, and fungi functioning as pumps, channels, or activators of enzymatic domains, with no G~protein involved. Despite their functional diversity, Type~I opsins share a highly conserved fold. They usually bind retinal in the all-\textit{trans} configuration and light triggers its isomerization to 13-\textit{cis}. The first member to be characterized was bacteriorhodopsin, discovered in 1971 in \textit{Halobacterium salinarum}\cite{oesterhelt1971}, which pumps protons across the membrane. The family has since expanded dramatically. Channelrhodopsins, which open cation channels in response to light, enabled optogenetics---the use of light to control genetically targeted cells---and transformed neuroscience\cite{nagel2003,deisseroth2015}. 

Type~II opsins are GPCRs found in animals\cite{shichida2009}. Retinal binds in the 11-\textit{cis} configuration, and light triggers isomerization to all-\textit{trans}. The conformational change activates an intracellular G~protein, initiating a signaling cascade; the protein itself does not pump ions or open channels. The first member to be structurally characterized was bovine rhodopsin, the photoreceptor in rod cells that enables dim-light vision\cite{palczewski2006}. Cone opsins, which mediate color vision, are closely related and exist in multiple spectral variants\cite{nathans1987}. The spectral tuning of these opsins determines which wavelengths each type detects.

These two families represent an example of convergent evolution\cite{ernst2014}---two separate protein lineages faced the same problem of detecting light and arrived at the same solution: binding retinal via a protonated Schiff base and using light-triggered isomerization to drive protein conformational changes.

\begin{figure}[H]
  \centering
  \includegraphics[width=\textwidth]{\dir/fig_type1_type2_comparison.jpg}
  \caption[Structural comparison of Type~I and Type~II opsins]{%
    Structural comparison of Type~I and Type~II opsins.
    (A)~Bovine rhodopsin (PDB: 1U19), an animal opsin (Type~II/GPCR), shown in transparent cartoon with the retinal chromophore (rust) and Schiff base lysine K296 (green).
    (B)~Bacteriorhodopsin (PDB: 1C3W), a microbial rhodopsin (Type~I), shown in the same style.
    (C,D)~Binding pocket close-ups for bovine rhodopsin and bacteriorhodopsin, respectively, showing all residues within \SI{6}{\angstrom} of retinal as gray sticks.}
  \label{fig:type1-type2-comparison}
\end{figure}

This shared solution has an important consequence. Both families use the same chromophore in geometrically similar binding pockets. In bacteriorhodopsin, retinal binds to lysine~216 (K216) on TM7\cite{luecke1999}, while in bovine rhodopsin, retinal binds to lysine~296 (K296) on TM7\cite{palczewski2000}. In both cases, the chromophore sits in a pocket formed by transmembrane helices, the Schiff base is protonated, and a negatively charged residue---the counterion---sits nearby to stabilize the positive charge (\autoref{fig:type1-type2-comparison}).

Spectral properties of opsins such as $\lambda_{\text{max}}$ (the wavelength of maximum absorbance) depend on the residues surrounding retinal. The same ligand sits in structurally analogous pockets across both families, and the same physics governs how those pockets modulate its electronic properties. This means that any retinal binding domain, regardless of which protein fold surrounds it, can in principle be studied from the same perspective: how does the local environment shape retinal's absorption? If this perspective holds, data from both opsin families---and even non-opsin retinal binders---could inform a single model of spectral tuning.

Standardized positional annotation are needed to make this concrete. Within a family, knowing which residue in one protein corresponds to which in another enables systematic comparison. Type~II opsins, as GPCRs, benefit from the existence of a generic numbering scheme, the Ballesteros--Weinstein system, and thus researchers can compare binding pocket residues across animal opsins using standardized positions. However, Type~I opsins lack an equivalent reference system. Currently, the field resorts to using bacteriorhodopsin as a reference, numbering positions in other proteins by their alignment to bR. But the sequences of microbial rhodopsins have diverged too far for reliably identifying equivalent positions across the whole superfamily by sequence alignment.


\section{Spectral Tuning}
\label{sec:intro-spectral-tuning}

Every opsin has a characteristic absorption spectrum, and the peak of that spectrum, the wavelength at which the protein absorbs light most strongly, is called $\lambda_\text{max}$. This parameter determines what color of light activates the opsin. An opsin with $\lambda_\text{max}$ at \SI{480}{\nano\metre} responds best to blue light, one at \SI{530}{\nano\metre} responds to green, and one at \SI{620}{\nano\metre} responds to red. Across both opsin families, $\lambda_\text{max}$ ranges from approximately \SI{350}{\nano\metre} (ultraviolet) to \SI{650}{\nano\metre} (far-red), a span of \SI{300}{\nano\metre} achieved using the same chromophore (\autoref{fig:charge-distribution}).

\begin{figure}[H]
  \centering
  \includegraphics[width=0.7\textwidth]{\dir/fig_charge_distribution.jpg}
  \caption[Charge distribution along the retinal chromophore]{%
    Charge distribution along the retinal chromophore. The retinal of bovine rhodopsin (PDB: 1U19) is shown as sticks within a transparent Gaussian surface envelope, colored by a point-charge gradient from the protonated Schiff base nitrogen (red, positive) to the $\beta$-ionone ring (blue, neutral). Positive charge density decreases along the conjugated polyene chain, and binding pocket residues modulate this distribution---shifting the energy gap for light absorption and thus determining $\lambda_\text{max}$.}
  \label{fig:charge-distribution}
\end{figure}


Spectral properties matter for applications. Optogenetics---the use of light to control genetically targeted cells---depends on opsins with known $\lambda_\text{max}$ values, because the wavelength of the stimulus determines which opsin responds. Red-shifted opsins ($\lambda_\text{max} > \SI{580}{\nano\metre}$) are particularly valuable: red light penetrates tissue more deeply than blue, enabling activation of cells in deeper brain regions or thicker tissue. Blue-shifted opsins ($\lambda_\text{max} < \SI{450}{\nano\metre}$) enable spectral multiplexing, allowing multiple opsins with different $\lambda_\text{max}$ values to be activated independently in the same tissue. Engineering opsins with specific spectral properties requires understanding the molecular basis of what determines $\lambda_\text{max}$.

The chromophore itself provides only part of the answer. Retinal (aldehyde) in solution absorbs at approximately \SI{380}{\nano\metre}\cite{honig1979}, near ultraviolet light, yet when bound to different opsins, absorption spans from the ultraviolet to the far-red. The protein environment tunes the chromophore, and understanding how this large spectral range is achieved using a single chromophore requires examining the local environment around the retinal.

Retinal is a polyene, a chain of alternating single and double bonds. The electrons in these double bonds are delocalized across the chain, forming a $\pi$-electron system. Light absorption promotes an electron from a lower-energy state to a higher-energy state. The energy gap between these states determines the wavelength of light absorbed. A smaller gap means lower energy, which corresponds to longer wavelengths (red shift). A larger gap means higher energy, which corresponds to shorter wavelengths (blue shift).

The protonation state of the Schiff base is critical: protonated opsins absorb in the visible range (above \SI{400}{\nano\metre}), while deprotonated opsins absorb in the ultraviolet (below \SI{400}{\nano\metre}), as seen in UV-sensitive SWS1 cone opsins\cite{yokoyama2008}. How the protein environment stabilizes the positive charge of the protonated Schiff base affects the distribution of electron density across the retinal's conjugated system, altering the energy gap and thereby determining $\lambda_\text{max}$.

Beyond the counterion, other binding pocket residues contribute through hydrogen bonds, local dipoles, and dispersive interactions with retinal's $\pi$-system. The cumulative effect of all these interactions determines $\lambda_\text{max}$.

Small changes in the binding pocket can produce large spectral shifts. Researchers interested in optogenetics exploit this to engineer opsins with novel spectral properties. The D85N mutation in bacteriorhodopsin, for example, replaces the negatively charged counterion with a neutral asparagine, producing a red shift of approximately \SI{25}{\nano\metre}\cite{mogi1988}. In bovine rhodopsin, the equivalent mutation E113Q has a different outcome: removing the counterion destabilizes the protonated Schiff base, causing it to deprotonate and shifting absorption from \SI{500}{\nano\metre} to approximately \SI{380}{\nano\metre}---a blue shift into the UV. The counterion thus serves a dual role: it tunes $\lambda_\text{max}$ electrostatically and maintains the protonation state required for visible-light absorption. Other binding pocket residues contribute individual shifts of \SIrange{5}{30}{\nano\metre}.

Beyond understanding, researchers seek prediction. The ability to predict $\lambda_\text{max}$ from sequence alone would enable rapid screening of novel sequences and guide engineering efforts. Family-specific methods exist. OPTICS\cite{frazer2024} predicts $\lambda_\text{max}$ for Type~II animal opsins using sequence features and phylogenetic information trained on the Visual Physiology Opsin Database (VPOD), achieving approximately \SI{6}{\nano\metre} mean absolute error. For Type I microbial opsins, Inoue et~al.\cite{inoue2020} developed a LASSO regression model using binding pocket residue identities, achieving \SI{8}{\nano\metre} MAE. RhoMax\cite{sela2024} uses a graph neural network on Type~I microbial opsin binding pocket structures to achieve \SI{7}{\nano\metre} MAE.

These methods are family-specific by construction. OPTICS and Inoue et~al.\ rely on multiple sequence alignments within their respective families, and RhoMax was trained exclusively on Type~I data. The scientific question, however, is not family-specific. Given a retinal binding pocket---regardless of which protein fold surrounds it---how does it modulate the spectral properties of its ligand? This is the central question of the thesis: a unified model that predicts opsin spectral properties across protein folds.


\section{Thesis Contributions}
\label{sec:intro-contributions}

LAMBDA (Light Absorption Modeling through Binding Domain Analysis) is the central contribution of this thesis. It is a spectral property prediction model for retinal-binding proteins not restricted to Type~I or Type~II opsins individually, but trained across both families and even a non-opsin retinal binder (human cellular retinol binding protein II; hCRBPII). LAMBDA represents binding pockets as graphs, enabling a single model to learn from any retinal binding domain regardless of the surrounding protein fold. Trained on 2,120 sequences, LAMBDA achieves \SI{5}{\nano\metre} mean absolute error on Type~II opsins and \SI{7}{\nano\metre} on Type~I. LAMBDA was applied to 47,700 opsin sequences to create the Opsin Atlas, a dataset of predicted spectral properties spanning the known diversity of both families.

MOGRN (Microbial Opsin Generic Residue Numbering) standardizes positional annotation for microbial rhodopsins by introducing a structure-based numbering system anchored to the conserved retinal-binding pocket. Validated against 129 structures spanning the functional diversity of microbial rhodopsins, MOGRN provides a common coordinate system for analyzing binding pocket residues across the Type~I family. LAMBDA uses MOGRN positions for Type~I opsins and Ballesteros--Weinstein positions for Type~II opsins to construct binding pocket graphs from sequence alone.

ProtOS is a Python framework for protein data management. It integrates the handling of protein sequences, structures, embeddings, and annotations through a consistent processor architecture, providing the infrastructure needed to compute embeddings, annotate GRN positions, and construct pocket graphs at scale. ProtOS-MCP adds a natural language interface through the Model Context Protocol, making these capabilities accessible to structural biologists who do not write code.
\chapter{Introduction}
\label{ch:introduction}

\section{A Brief History of Protein Structures}
\label{sec:intro-history}

For most of history, proteins were invisible. Biochemists knew they existed, knew they catalyzed reactions and transmitted signals, but could not see them. X-ray crystallography changed this. In 1958, John Kendrew solved the structure of myoglobin\cite{kendrew1958}. For the first time, researchers could see how a protein folded, a chain of amino acids wrapped into a compact globule, with a heme group nestled in a hydrophobic pocket where oxygen binds. The structure revealed why myoglobin worked. Myoglobin is a soluble protein, floating freely in the cytoplasm.

Membrane proteins proved far harder. These proteins (channels, pumps, and receptors embedded in cell membranes) have hydrophobic surfaces that contact lipids and also hydrophilic surfaces exposed to water on the intra- or extracellular side of the membrane. Removed from the membrane, they aggregate and aggregated proteins cannot form the ordered crystals that X-ray crystallography requires. Alas, membrane protein structures remained out of reach for decades. Yet these proteins handle many interactions between a cell and its environment, making them essential targets for understanding biology.

Despite these challenges, the first view of a membrane protein came in 1975. Richard Henderson and Nigel Unwin used electron microscopy to image bacteriorhodopsin, a light-driven proton pump from \textit{Halobacterium salinarum}\cite{henderson1975}. The resolution was \SI{7}{\angstrom}, too coarse to resolve individual atoms, but the map showed seven rod-like densities spanning the membrane. Bacteriorhodopsin was built from seven transmembrane helices. Bacteriorhodopsin is a microbial protein, but the seven-helix architecture also appeared in an important class of animal membrane proteins.

In 1993, Gebhard F.X.\ Schertler, Claudio Villa, and Richard Henderson published a projection map of bovine rhodopsin, the light-sensitive protein of vertebrate eyes and the first GPCR amenable to structural analysis\cite{schertler1993}. GPCRs---G~protein coupled receptors---got their names from their effector proteins, the G~proteins. The GPCR activates the G~proteins inside the cell which leads to a signalling cascade. In the case of bovine rhodopsin, a single photon can lead to a structural change in the bovine rhodopsin starting that cascade---a very effective approach cells have to amplify tiny signals into large cellular responses. Back to their first projection map: the \SI{9}{\angstrom} map showed that rhodopsin, like bacteriorhodopsin, contained seven transmembrane helices.

Then in 2000, Krzysztof Palczewski solved bovine rhodopsin at \SI{2.8}{\angstrom}, the first GPCR at atomic resolution\cite{palczewski2000}, confirming the seven-helix architecture. This high resolution revealed how the retinal, the light-absorbing chromophore, sits in the transmembrane core.

Later in 2007, Brian Kobilka's group solved the $\beta_2$-adrenergic receptor\cite{rasmussen2007}, showing that other GPCRs could be crystallized. Kobilka shared the 2012 Nobel Prize with Robert Lefkowitz for their work.

Crystallography requires proteins to form ordered crystals, a barrier that excludes many membrane proteins. Cryo-electron microscopy, another method to elucidate protein structures, led to many more structures being solved. The technique flash-freezes proteins in thin ice, preserving their conformations, but for decades cryo-EM produced only blurry images, useful for overall shape but not atomic detail. The breakthrough came around 2013 with direct electron detectors, which dramatically improved signal quality, and new algorithms that could align and average millions of particle images. Resolution improved from \SIrange{10}{20}{\angstrom} to \SI{3}{\angstrom}. Even protein complexes that would not crystallize (large, flexible, or transient assemblies) could be imaged. Henderson shared the 2017 Nobel Prize in Chemistry with Jacques Dubochet and Joachim Frank for this work.

Together these two methods shaped structural biology. The largest experimental database for protein structures, the Protein Data Bank grew rapidly, from roughly 1,000 structures in 1993 to over 200,000 by 2024, with GPCR structures alone rising from fewer than 5 in 2007 to over 700 today.\worktodo{Fact-check: verify ``over 200,000 by 2024'' for PDB total and ``over 700'' for GPCR structures against current database counts.}

Structure determination investigates one protein at a time. Sequencing opened a window onto the full diversity of the protein world. Sanger sequencing gave way to genomics, then metagenomics, the sequencing of DNA extracted directly from environmental samples. UniProt now contains over 200 million protein sequences, many from ocean surveys, soil samples, and gut microbiomes. For some protein families, metagenomics revealed unexpected abundance. Microbial rhodopsins are a great example. What had seemed a peculiar protein of archaea turned out to be abundant everywhere.\worktodo{Author note~[a]: expand on microbial rhodopsin abundance claim.} Discovered in 2000 from Pacific Ocean samples\cite{beja2000}, they are now recognized as among the most abundant proteins on Earth, and metagenomics has revealed thousands of rhodopsin sequences across bacteria, archaea, algae, and fungi.

Unsurprisingly, sequence diversity far exceeds structural coverage using experimental methods. In 2020, DeepMind's AlphaFold2 took first place in the annual protein structure prediction competition, demonstrating that protein structures could be predicted from sequence---with a huge leap in accuracy\cite{jumper2021}. Within two years, DeepMind and EMBL-EBI released predicted structures for over 200 million proteins\cite{varadi2022}. And structure prediction continues to advance. New systems like Boltz\cite{wohlwend2024} now predict proteins bound to their ligands---small molecules, cofactors, and interaction partners---enabling computational analysis of binding pockets without requiring experimental structures of each complex.

Alongside structure prediction, other deep learning methodologies such as protein language models have transformed how researchers extract information from amino acid sequences. These models train on millions of protein sequences, learning statistical patterns that reflect evolutionary constraints and structural relationships. Models like ESM-2 and Ankh produce embeddings---dense numerical vectors for each amino acid in a sequence. A 300-residue protein yields 300 vectors, each capturing what the model has learned about that residue in its sequence context. Averaging these vectors across the sequence produces a single representation of the entire protein. Two proteins with diverged sequences but similar functions often have similar embeddings, even when their pairwise sequence identity is low. This property makes embeddings valuable as input features for machine learning models that predict protein properties.

With these advances, the field saw an explosion in data abundance, and every known protein sequence now has a readily available predicted structure.\worktodo{Fact-check: this overstates coverage. AlphaFold~DB covers ${\sim}$200M of $>$250M UniProt sequences, and many metagenomic sequences lack predictions. Soften.}

Yet predicted structures lack context. AlphaFold predictions come with confidence scores and little else, lacking experimental conditions or functional annotations. Comparing predictions to experimental structures requires careful handling of coordinate systems and numbering conventions. Annotating hundreds or tens of thousands of structures needs automation. The bottleneck has shifted from generating data to understanding it. For researchers studying membrane protein families like GPCRs and microbial rhodopsins, this shift has practical consequences. These families now have hundreds of structures and are targets for drug discovery and biotechnology, yet systematic studies across their members remains difficult.


\section{Studying Proteins}
\label{sec:intro-studying}

Understanding protein function requires more than accumulating structures; it requires comparing them. Data abundance enables diverse approaches to this challenge. For protein superfamilies---large, evolutionarily related groups sharing a common fold---one particularly powerful approach is systematic comparison across members. Hundreds of related proteins, each with slightly different sequences and functions, create natural experiments. Examining which residues vary and which are conserved reveals the molecular basis of functional differences.

Such comparison requires knowing which positions are equivalent---position~85 in one protein may correspond to position~113 in another. For closely related proteins, sequence alignment resolves this, but distant superfamily members may share too little sequence identity for reliable alignment. Structure-based approaches are more robust but require solved structures for both proteins. For GPCRs researchers tackled this problem in 1995. Ballesteros and Weinstein introduced a numbering system that assigns each position a two-part identifier consisting of the helix number and position relative to the most conserved residue on that helix\cite{ballesteros1995}. The format is X.50 for the reference residue on helix~X, with other positions numbered relative to this anchor. Position~3.32 indicates helix~3, at the position numbered~32 (eighteen residues before the reference position~50). This scheme translates across the entire GPCR superfamily, roughly 800 members in the human genome. Scientists adopted the numbering system broadly, and GPCRdb built on this foundation\cite{kooistra2021}. This database integrates sequences, structures, ligands, and mutations using Ballesteros--Weinstein positions as the common coordinate system, enabling queries such as identifying the residue at position~3.32 across all aminergic receptors within seconds.

This numbering system shaped how an entire field communicates. When a researcher reports that a mutation at position~3.32 affects ligand binding, every GPCR researcher immediately understands where that position is and can relate the finding to their own work. With standardized positions, binding sites can be compared across receptors directly. If a drug binds at position~3.32 in one receptor, researchers can immediately ask what residue sits at~3.32 in related receptors and whether side effects might arise from off-target binding. Functional insights accumulate across laboratories because everyone uses the same coordinate system. The scheme works because GPCRs share a conserved seven-transmembrane fold---the helical architecture is stable enough across the superfamily that helix-relative numbering remains meaningful even between distantly related members. The GPCR field illustrates what a common coordinate system enables. Another seven-transmembrane protein family---the microbial rhodopsins---lacked such standardization entirely.


\section{Opsins}
\label{sec:intro-opsins}

This thesis studies opsins. They are a family of seven-transmembrane proteins that become light-sensitive upon binding retinal as their chromophore. They exist across all domains of life (archaea, bacteria, algae, fungi, and animals) where they perform diverse functions including pumping ions, opening channels, sensing light direction, and activating signaling cascades.

Opsins bind retinal, a vitamin~A derivative, through a covalent Schiff base linkage to a conserved lysine residue---a bond formed between retinal's aldehyde group and the lysine's amino group. Strictly, opsin refers to the apoprotein alone, while rhodopsin refers to the protein with retinal bound; in practice, the terms are used interchangeably.\worktodo{Author note~[b]: clarify opsin vs.\ rhodopsin terminology.} When light hits the retinal, it isomerizes, triggering a conformational change in the protein that serves as the signal. The protein environment around the retinal tunes which wavelengths of light are absorbed. All opsins share the same basic mechanism: retinal isomerization coupled to conformational change.

Despite sharing a chromophore, opsins fall into two evolutionarily completely distinct families.

Type~I opsins, the microbial rhodopsins, are found in archaea, bacteria, algae, and fungi. These are not GPCRs and share no sequence homology with type~II (animal opsins), adopting a distinct fold despite sharing a seven-transmembrane helix architecture. Retinal usually binds in the all-\textit{trans} configuration, and light triggers isomerization to 13-\textit{cis}. The protein responds directly by pumping ions, opening a channel, or activating an enzymatic domain, with no G~protein involved. The founding member\worktodo{Author note~[c]: confirm ``founding member'' is accurate.} is bacteriorhodopsin, discovered in 1971 in \textit{Halobacterium salinarum}\cite{oesterhelt1971}, which pumps protons across the membrane. The family has since expanded to include ion pumps with different specificities, light-gated channels, sensory receptors, and light-activated enzymes. Channelrhodopsins, which open cation channels in response to light, enabled optogenetics---the use of light to control genetically targeted cells---and transformed neuroscience\cite{nagel2003}. Despite this functional diversity, all Type~I opsins share the same architecture.

Type~II opsins are GPCRs found in animals.\worktodo{Author note~[d]: note exceptions for non-vertebrate animal opsins.} Retinal binds in the 11-\textit{cis} configuration, and light triggers isomerization to all-\textit{trans}. The conformational change activates a bound G~protein, initiating a signaling cascade; the protein itself does not pump ions or open channels. The founding member is bovine rhodopsin, the photoreceptor in rod cells that enables dim-light vision. Cone opsins, which mediate color vision, are closely related and exist in multiple spectral variants. The spectral tuning of these opsins determines which wavelengths each type detects.

Type~I and Type~II opsins are not homologous. They share no detectable sequence similarity, and their seven-transmembrane folds are structurally distinct. These two families represent nature evolving the same function---light sensitive proteins using retinal as a ligand---in analogous structures, not homologous ones. Two separate protein lineages faced the same problem of detecting light and arrived at the same solution, binding retinal via a protonated Schiff base (one that carries a positive charge) and using light-triggered isomerization to drive conformational change.\worktodo{Author note~[e]: consider if convergent evolution claim needs citation.}

This shared solution has an important consequence. Both families use the same chromophore in geometrically similar binding pockets. In bacteriorhodopsin, retinal binds to lysine~216 (K216) on helix~G, while in bovine rhodopsin, retinal binds to lysine~296 (K296) on helix~VII.\worktodo{Narrative: ``helix~G'' (type~I) and ``helix~VII'' (type~II) introduce two helix naming conventions. The rest of the thesis uses TM1--TM7 for both families. Consider standardizing to TM7 here.} In both cases, the chromophore sits in a pocket formed by transmembrane helices, the Schiff base is protonated, and a negatively charged residue (aspartate or glutamate, called the counterion) sits nearby to stabilize the positive charge (\autoref{fig:type1-type2-comparison}). Despite independent origins, both families converged on shared geometry because that geometry solves a physical problem, namely tuning retinal's electronic properties to absorb light at specific wavelengths.

\begin{figure}[tb]
  \centering
  % \includegraphics[width=\textwidth]{\dir/figure_1_1.pdf}
  \caption[Structural comparison of Type~I and Type~II opsins]{%
    Structural comparison of Type~I and Type~II opsins.
    (A)~Bacteriorhodopsin (PDB: 1C3W), a microbial rhodopsin (Type~I), shown in transparent cartoon with the retinal chromophore (rust) and Schiff base lysine K216 (green).
    (B)~Bovine rhodopsin (PDB: 1U19), an animal opsin (Type~II/GPCR), shown in the same style.
    (C,D)~Binding pocket close-ups for bacteriorhodopsin and bovine rhodopsin, respectively, showing all residues within \SI{6}{\angstrom} of retinal as gray sticks.}
  \label{fig:type1-type2-comparison}
\end{figure}

If local environment determines function, and both families have similar local environments, then functional principles might transfer across families. This premise enables cross-family analysis. The spectral properties of an opsin, meaning the wavelengths of light it absorbs, depend on the residues surrounding the chromophore, and if those residues occupy similar positions in both families, the rules governing spectral tuning might be shared.

Such analysis requires consistent positional annotation, knowing which residues in one protein correspond to which in another. Type~II opsins, as GPCRs, benefit from Ballesteros--Weinstein numbering, and researchers can compare binding pocket residues across animal opsins using standardized positions. Not for lack of trying. Early efforts used bacteriorhodopsin as a reference, numbering positions in other proteins by their alignment to bR. But microbial rhodopsins have diverged too far for sequence alignment to reliably identify equivalent positions across the full superfamily. A more recent approach defined positions based on conserved hydrogen bond networks, but these networks differ between functional classes---what holds for bacteriorhodopsin does not generalize to halorhodopsins, channelrhodopsins, or enzyme rhodopsins. The problem remains unsolved.


\section{Spectral Tuning}
\label{sec:intro-spectral-tuning}

Every opsin has a characteristic absorption spectrum, and the peak of that spectrum, the wavelength at which the protein absorbs light most strongly, is called $\lambda_\text{max}$. This parameter determines what color of light activates the opsin. An opsin with $\lambda_\text{max}$ at \SI{480}{\nano\metre} responds best to blue light, one at \SI{560}{\nano\metre} responds to green, and one at \SI{620}{\nano\metre} responds to red. Across the opsin superfamily, $\lambda_\text{max}$ ranges from approximately \SI{350}{\nano\metre} (ultraviolet) to \SI{650}{\nano\metre} (far-red), a span of \SI{300}{\nano\metre} achieved using the same chromophore (\autoref{fig:charge-distribution}).

\begin{figure}[tb]
  \centering
  % \includegraphics[width=\textwidth]{\dir/figure_1_2.pdf}
  \caption[Charge distribution along the retinal chromophore]{%
    Charge distribution along the retinal chromophore. The retinal of bovine rhodopsin (PDB: 1U19) shown as sticks within a transparent Gaussian surface envelope, colored by a point-charge gradient from the protonated Schiff base nitrogen (red, positive) to the $\beta$-ionone ring (blue, neutral). Positive charge density decreases along the conjugated polyene chain, and binding pocket residues modulate this distribution---shifting the energy gap for light absorption and thus determining $\lambda_\text{max}$.}
  \label{fig:charge-distribution}
\end{figure}

Spectral properties matter for applications. In optogenetics, controlling which wavelengths activate an opsin enables precise manipulation. Red-shifted opsins ($\lambda_\text{max} > \SI{600}{\nano\metre}$) allow deeper tissue penetration because red light travels further through tissue than blue. Blue-shifted opsins ($\lambda_\text{max} < \SI{450}{\nano\metre}$) enable spectral separation,\worktodo{Author note~[f]: define ``spectral separation'' more precisely.} allowing multiple opsins with different $\lambda_\text{max}$ values to be activated independently in the same tissue. Engineering opsins with specific spectral properties requires understanding the molecular basis of what determines $\lambda_\text{max}$.

The chromophore itself provides only part of the answer. Retinal in solution absorbs at approximately \SI{380}{\nano\metre}\worktodo{Fact-check: \SI{380}{\nano\metre} is for retinal aldehyde. A protonated retinal Schiff base in solution absorbs at ${\sim}$\SI{440}{\nano\metre}. Clarify which species is meant.}, near ultraviolet light, yet when bound to different opsins, absorption can shift by over \SI{200}{\nano\metre}, ranging from below \SI{350}{\nano\metre} to over \SI{600}{\nano\metre}. The protein environment tunes the chromophore, and understanding how evolution achieves over \SI{250}{\nano\metre} of spectral range using a single chromophore requires examining the local environment around the retinal.

Retinal is a polyene, a chain of alternating single and double bonds. The electrons in these double bonds are delocalized across the chain, forming a $\pi$-electron system. Light absorption promotes an electron from a lower-energy state to a higher-energy state. The energy gap between these states determines the wavelength of light absorbed. A smaller gap means lower energy, which corresponds to longer wavelengths (red shift). A larger gap means higher energy, which corresponds to shorter wavelengths (blue shift).

In opsins, retinal attaches to the protein through a Schiff base linkage to a conserved lysine (\eg K296 in bovine rhodopsin, K216 in bacteriorhodopsin). In most opsins, this Schiff base is protonated, meaning it carries a positive charge. The protonation state is critical: protonated opsins absorb in the visible range (above \SI{400}{\nano\metre}), while deprotonated opsins absorb in the ultraviolet (below \SI{400}{\nano\metre}), as seen in UV-sensitive SWS1 cone opsins. The protonated Schiff base creates a positive charge in the binding pocket. How the protein environment stabilizes this charge affects the distribution of electron density across the retinal's conjugated system, altering the energy gap between ground and excited states and thereby determining $\lambda_\text{max}$.

In many opsins, a negatively charged residue (aspartate or glutamate) sits near the protonated Schiff base and stabilizes the positive charge. This residue is called the counterion. In bacteriorhodopsin, the counterion is D85. In bovine rhodopsin, it is E113. A counterion that stabilizes the positive charge more strongly localizes it near the Schiff base. This increases the energy gap and shifts absorption to shorter wavelengths (blue shift). Conversely, a counterion allowing the positive charge to delocalize along the polyene chain decreases the energy gap and shifts absorption to longer wavelengths (red shift).

Other residues in the binding pocket contribute too. Polar and charged residues shape the electrostatic environment through hydrogen bonds, local dipoles, and long-range Coulombic interactions. Aromatic residues influence retinal through both steric constraints and dispersive interactions with its $\pi$-system. The cumulative effect of all these interactions determines $\lambda_\text{max}$.

Single mutations at key positions can produce large spectral shifts. The D85N mutation in bacteriorhodopsin replaces the negatively charged aspartate counterion with a neutral asparagine, producing a red shift of up to \SI{40}{\nano\metre}.\worktodo{Fact-check: verify the D85N shift magnitude and whether E113Q in bovine rhodopsin ``produces a similar effect.''} The equivalent mutation in bovine rhodopsin, E113Q, produces a similar effect. The counterion is a major determinant of $\lambda_\text{max}$, though not the only one; other binding pocket residues contribute individual shifts of \SIrange{10}{30}{\nano\metre}.

These examples illustrate that understanding spectral tuning requires identifying which residues matter and why, but also how they vary across proteins. Studying this systematically requires comparing binding pocket residues, which circles back to standardization: without shared positional coordinates, systematic comparison across the superfamily is impossible. Beyond comparison, researchers seek prediction. The ability to predict $\lambda_\text{max}$ from sequence alone would enable rapid screening of metagenomic discoveries and guide engineering efforts. Family-specific methods exist for this purpose. OPTICS\cite{frazer2025} predicts $\lambda_\text{max}$ for type~II opsins using sequence features and phylogenetic information trained on the VPOD dataset\cite{davis2024}, achieving approximately \SI{5.5}{\nano\metre} mean absolute error, but working only within the GPCR family. For microbial opsins, Inoue et~al.\cite{inoue2021} developed a LASSO regression model using binding pocket residue identities, achieving \SI{7.8}{\nano\metre} MAE on their test set. RhoMax\cite{gerstenbruch2024}, the most recent approach, uses a graph neural network on type~I opsin binding pocket structures to achieve \SI{6.8}{\nano\metre} MAE.

These methods are family-specific by construction. OPTICS and Inoue et~al.\ both rely on multiple sequence alignments within their respective families, using each alignment position as a regression feature---an approach that depends on reliable alignment and therefore cannot extend across families. RhoMax sidesteps alignment by operating on structures directly but requires a solved or predicted structure for every query protein and was trained exclusively on Type~I data.

Existing tools can predict $\lambda_\text{max}$ within a family but not across families, and each depends on infrastructure---reliable alignments or available structures---that is family-specific. Yet the underlying question is not family-specific. If spectral properties are determined by the local environment around the retinal, then the relevant unit of analysis is not ``a Type~I opsin'' or ``a Type~II opsin'' but a retinal binding pocket. Given any retinal binding pocket, regardless of which protein scaffold surrounds it, what are its spectral properties? Framed this way, the Type~I / Type~II distinction becomes irrelevant to the prediction problem---what matters is the geometry and chemistry of the pocket itself. Two research questions follow.

First, how can standardized positions be established across microbial rhodopsins? The systematic comparison that enabled GPCR research requires a common coordinate system. Microbial rhodopsins need the same.

Second, can a retinal binding pocket alone predict spectral properties? If the local chromophore environment determines $\lambda_\text{max}$, a model should be able to take any retinal binding pocket as input and predict absorption---regardless of whether that pocket belongs to a proton pump, a channel, or an animal photoreceptor.

Addressing both questions requires capabilities that current tools do not provide, specifically linking sequences, structures, and functional properties across databases while maintaining consistent positional annotation.


\section{Fragmentation in Bioinformatics}
\label{sec:intro-fragmentation}

Answering these questions requires combining protein sequences from UniProt, structures from the PDB and AlphaFold~DB, measured $\lambda_\text{max}$ values from the literature, embeddings from protein language models, and annotations such as binding pocket definitions. No single database contains all of this. Researchers must gather data from multiple sources, convert between formats, and track which identifiers refer to the same protein---UniProt calls it P02945, the PDB calls the structure 1C3W, and AlphaFold calls its prediction AF-P02945-F1.

Tools exist for individual tasks. Biopython handles sequence retrieval and alignment, dedicated suites like the Schr\"odinger platform provide structure analysis and molecular modeling, and protein language models produce embeddings through their own pipelines. But these tools were not designed to work together. A typical analysis---comparing spectral properties across 100 microbial rhodopsins---requires downloading sequences, fetching structures or predictions, collecting $\lambda_\text{max}$ values from the literature, aligning sequences, mapping sequence positions to structural coordinates, extracting binding pocket residues, and correlating those features with absorption. Each step uses a different tool and a different format. The glue code connecting them dominates the work and gets rewritten for each project.

This fragmentation limits both communities that study opsins. Experimentalists who measure $\lambda_\text{max}$ values cannot easily compare them to structural features without programming expertise. Bioinformaticians can perform the analysis but spend disproportionate effort on data engineering that contributes nothing to scientific insight. The barrier is tooling, not ideas.

Establishing standardized positions across microbial rhodopsins requires assembling and aligning structures from multiple sources. Testing whether a retinal binding pocket alone predicts spectral properties requires linking thousands of sequences to their $\lambda_\text{max}$ values, embeddings, and structural features. These are not extraordinary requirements, but no existing framework addresses them.


\section{Thesis Contributions}
\label{sec:intro-contributions}

Microbial rhodopsins lack the standardized positional annotation that enabled systematic GPCR comparison. No method predicts spectral properties from a retinal binding pocket regardless of which protein family it belongs to. And existing tools do not compose into workflows that link sequences, structures, and functional properties while maintaining consistent identity. This thesis addresses all three through an integrated system.

ProtOS provides the data infrastructure that underlies the other contributions. Protein research requires linking sequences, structures, functional properties, and computational representations across databases that use different identifiers, formats, and conventions. ProtOS manages these relationships through a processor architecture that handles specific data types (sequences, structures, embeddings, properties, residue annotations) through a consistent interface. The framework enables workflows that span data modalities without the glue code that currently dominates computational protein analysis. ProtOS-MCP extends this through the Model Context Protocol, enabling natural language access to protein data operations---an experimentalist can ask which residues contact retinal and receive an answer without writing code.

MOGRN (Microbial Opsin Generic Residue Numbering) establishes standardized positional annotation for microbial rhodopsins. Overall sequence identity across major classes falls below 15\%, yet the retinal-binding pocket geometry remains conserved. MOGRN introduces a structure-based numbering system anchored to the conserved retinal-binding site, following the Ballesteros--Weinstein convention used for GPCRs. The Schiff base lysine becomes position~7.50 by definition; other positions are numbered relative to helix-specific anchors. Validated against 129 structures spanning the functional diversity of microbial rhodopsins, the system provides a common coordinate system that enables systematic comparison across the superfamily.

LAMBDA (Light Absorption Modeling through Binding Domain Analysis) tests whether shared binding pocket geometry enables cross-family learning for spectral prediction. Existing methods work within single opsin families but do not transfer across the type~I / type~II divide. LAMBDA represents binding pockets as graphs, enabling a single model to learn spectral property prediction for any retinal binding protein. I also show that LAMBDA works with only protein sequence inputs if there is a standardized positional annotation: MOGRN for type~I, Ballesteros--Weinstein for type~II. Trained on 2,120\worktodo{Author note~[g]: verify exact training set size.} sequences spanning both opsin families and hCRBPII (a lipocalin fold), LAMBDA achieves state-of-the-art accuracy: \SI{5.18}{\nano\metre} mean absolute error on type~II opsins and \SI{6.86}{\nano\metre} on type~I\worktodo{CRITICAL consistency: these numbers (2,120 training / \SI{5.18}{\nano\metre} / \SI{6.86}{\nano\metre}) contradict \autoref{ch:discussion} which says 2,054 / \SI{4.08}{\nano\metre} / \SI{6.27}{\nano\metre}. Reconcile across all chapters.}, while providing cross-family capability that existing methods lack. I used ProtOS to create a comprehensive dataset of sequences containing over 47,700 sequences, that I named the Opsin Atlas. I predicted the spectral properties for all its members, enabling systematic identification of candidates with desired spectral properties.

These contributions integrate into a system. ProtOS manages the data that MOGRN annotates and LAMBDA consumes. The Opsin Atlas exists because ProtOS could route 47,700 sequences through GRN annotation and spectral prediction.
\chapter{ProtOS-MCP --- Talk to Your Proteins}
\label{ch:protos-mcp}

\chapterabstract{During my PhD, AI went from barely functional to a daily coding companion. Coding is about solving problems, not typing---but the barrier to entry remains real. ProtOS is powerful, yet the future lies in making advanced computational frameworks accessible to any structural biologist or bioinformatician. MCP is where lab scientists and computational researchers meet.}

ProtOS requires Python, which limits access for researchers who cannot code. An experimentalist who asks ``which residues contact retinal in this structure?'' understands what they want; they cannot get it without a programmer. ProtOS-MCP removes that barrier by exposing the same processors through natural language: a researcher describes what they want, and the system executes it. The contribution is not the natural language interface itself---large language models that call tools are not new---but the domain-specific infrastructure behind it.

The Model Context Protocol (MCP)\cite{anthropic2024mcp}, developed by Anthropic, standardizes how language models discover and invoke external tools through typed schemas. A language model generates text; it does not load structures, align sequences, or predict absorption spectra. MCP provides a structured interface through which the model can perform these operations by selecting the appropriate tool, populating its parameters, and interpreting the result.

\begin{figure}[H]
  \centering
  \includegraphics[width=\textwidth]{\dir/fig_mcp_architecture.png}
  \caption[ProtOS-MCP architecture]{%
    ProtOS-MCP architecture. A natural language question is routed through MCP tool calls to the local ProtOS server, which returns structured JSON. The language model synthesizes results into a response. All protein data remains local.}
  \label{fig:mcp-architecture}
\end{figure}

\autoref{fig:mcp-architecture} shows the ProtOS-MCP architecture. The researcher types a question in natural language. The language model interprets it and issues one or more MCP tool calls to the local ProtOS server. The server executes each call against the ProtOS installation and returns structured JSON. The model synthesizes the results into a natural language response. Critically, ProtOS-MCP runs locally---no protein data leaves the researcher's machine; only the conversational text passes through external servers.

To demonstrate the system, the rhodozyme sequence from \autoref{ch:rhodozyme} served as a starting point. Its retinal-binding domain is preserved from the parent rhodopsin, so spectral tuning through binding pocket mutations remains possible without disrupting the designed catalytic site. The question posed through conversation: can the rhodozyme's predicted absorption be shifted toward longer wavelengths by transplanting binding pocket residues from the human long-wave sensitive opsin?

The complete conversation (\autoref{sec:appendix-mcp-conversation}) proceeded in four turns, involving 20 tool calls across five processors. No code was written. The model first reviewed its capabilities, then ingested the rhodozyme FASTA file, predicted its absorption at \SI{496}{\nano\metre} for 11-\textit{cis} retinal using LAMBDA, and finally executed a multi-step redshift engineering workflow: comparing binding pockets at GRN positions, screening 20 single-point mutants, and performing a greedy cumulative walk that peaked at seven mutations with a predicted shift to \SI{514}{\nano\metre}. The model autonomously decomposed the complex fourth prompt into 11 sequential tool calls with correct dependencies---GRN annotation before pocket comparison before mutant design---without further prompting.

The value is not the resulting design, which awaits experimental validation. Four conversational turns compressed a multi-step spectral engineering analysis that would otherwise require detailed knowledge of the ProtOS API. The model also applied domain knowledge---such as identifying the protein family from the sequence---that was not programmed into the MCP tools. The limitations are equally clear: the workflow depends on the language model reasoning correctly about tool selection and parameter values, and a model that misunderstands one step propagates errors through the rest. ProtOS-MCP does not replace programming---analyses requiring custom logic or operations not exposed as tools still require code---but it lowers the barrier for the common case: a researcher with a biological question and data that fits the processor workflows.

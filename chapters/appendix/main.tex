\chapter{Supplementary Materials}
\label{ch:appendix}


\section{ProtOS-MCP Conversation Log}
\label{sec:appendix-mcp-conversation}

The complete ProtOS-MCP conversation used to redshift the rhodozyme (\autoref{ch:protos-mcp}) is reproduced on the following pages. The session proceeded in four turns involving 20 tool calls across five processors. No code was written.

\clearpage
\begin{figure}[H]
  \captionsetup{position=top}
  \caption*{(a)~Turn~1: orientation and capability review.}
  \vfill
  \centering
  \includegraphics[width=\textwidth,height=0.92\textheight,keepaspectratio]{\dir/fig_mcp_turn_1.png}
  \vfill
\end{figure}

\clearpage
\begin{figure}[H]
  \captionsetup{position=top}
  \caption*{(b)~Turn~2: data ingestion of rhodozyme FASTA.}
  \vfill
  \centering
  \includegraphics[width=\textwidth,height=0.92\textheight,keepaspectratio]{\dir/fig_mcp_turn_2.png}
  \vfill
\end{figure}

\clearpage
\begin{figure}[H]
  \captionsetup{position=top}
  \caption*{(c)~Turn~3: spectral prediction ($\lambda_\text{max} = \SI{496}{\nano\metre}$).}
  \vfill
  \centering
  \includegraphics[width=\textwidth,height=0.92\textheight,keepaspectratio]{\dir/fig_mcp_turn_3.png}
  \vfill
\end{figure}

\clearpage
\begin{figure}[H]
  \captionsetup{position=top}
  \caption*{(d)~Turn~4: rational redshift engineering through binding pocket comparison, single-mutant screening, and greedy cumulative walk (peak: \SI{514}{\nano\metre} at 7 mutations).}
  \vfill
  \centering
  \includegraphics[width=\textwidth,height=0.88\textheight,keepaspectratio]{\dir/fig_mcp_turn_4.png}
  \vfill
\end{figure}


\section{Workflow Benchmark Specifications}
\label{sec:appendix-workflow-benchmark}

The ProtOS-MCP workflow collection is stratified by difficulty based on the number of tools required, the number of processors involved, and the complexity of result interpretation.

\begin{table}[htb]
  \caption{Workflow benchmark difficulty criteria.}
  \label{tab:benchmark-difficulty}
  \centering
  \begin{tabularx}{\textwidth}{lllX}
    \toprule
    Level & Tools & Processors & Example \\
    \midrule
    Beginner & 1--2 & 1 & Load structure, list ligands \\
    Intermediate & 3--5 & 2--3 & Align structures, annotate with GRN \\
    Advanced & 5+ & 3+ & Multi-stage functional analysis \\
    \bottomrule
  \end{tabularx}
\end{table}

Each benchmark workflow is evaluated on: tool selection accuracy, parameter accuracy, execution order, result interpretation, and biological accuracy.


\section{Processor Data Storage Conventions}
\label{sec:appendix-storage}

ProtOS organizes data in a standardized directory structure. \autoref{tab:storage-conventions} describes the storage locations for each processor.

\begin{table}[htb]
  \caption{Processor data storage conventions.}
  \label{tab:storage-conventions}
  \centering
  \small
  \begin{tabularx}{\textwidth}{llX}
    \toprule
    Processor & Base Directory & File Formats \\
    \midrule
    Structure & \texttt{structure/} & \texttt{.cif}, \texttt{.pkl}, \texttt{.json} \\
    Sequence & \texttt{sequence/} & \texttt{.fasta}, \texttt{.aln}, \texttt{.mmseqs} \\
    GRN & \texttt{grn/} & \texttt{.yaml}, \texttt{.parquet} \\
    Embedding & \texttt{embedding/} & \texttt{.npy}, \texttt{.h5} \\
    Property & \texttt{property/} & \texttt{.parquet}, \texttt{.csv} \\
    Molecule & \texttt{molecule/} & \texttt{.sdf}, \texttt{.mol2}, \texttt{.json} \\
    \bottomrule
  \end{tabularx}
\end{table}

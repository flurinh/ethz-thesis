\chapter{MOGRN --- A Generic Residue Numbering System for Microbial Rhodopsins}
\label{ch:mogrn}

\chapterabstract{I believe MOGRN will have a significant impact on the optogenetics and microbial rhodopsin communities. For decades, GPCR researchers have shared a structural vocabulary---Ballesteros--Weinstein numbering---that made results immediately transferable across laboratories. Microbial rhodopsin research has lacked this. MOGRN provides it.}


\section{Introduction}
\label{sec:mogrn-introduction}

The work summarized here is presented in the manuscript ``A Generic Residue-Numbering System for Microbial Rhodopsins --- Unifying Structural Frameworks and Functional Mapping''\cite{hidber2026mogrn}, of which I am a co-first author (see Appendix \ref{app:a_generic_residue_numbering_system}).

The field of GPCRs solved the problem of generic numbering decades ago. The Ballesteros--Weinstein system\cite{ballesteros1995} assigns each position in the sequence a helix-relative identifier (helix number and offset from the most conserved residue) creating references that translate across the entire superfamily. When a researcher reports that a mutation at position~3.32 affects ligand binding, every GPCR researcher knows where that is. However, microbial rhodopsins---functionally and taxonomically diverse---had nothing equivalent, and laboratories developed ad~hoc conventions. Identity between major classes falls as low as 10--12\%\cite{degrip2022}, too low for reliable sequence alignment. Thus, a numbering system for microbial rhodopsins had to be anchored to structure.

We developed MOGRN (Microbial Opsin Generic Residue Numbering), anchored to the one feature all microbial rhodopsins share: the retinal-binding pocket. The system is defined by structural alignment of 129 structures spanning all major functional classes, producing a curated reference table, and was applied to approximately 40,000 sequences. The key parts of this work are shown below; the full methodology, results, and discussion are available in the linked manuscript.


\section{Summary}
\label{sec:mogrn-summary}

MOGRN is inspired by the Ballesteros--Weinstein system: each position receives a two-part identifier indicating the helix (1--7) and the position relative to a helix-specific anchor. The anchor on each helix is in our case the residue closest to retinal, designated X.50. The Schiff base-forming lysine on helix~7 becomes position~7.50 by definition. In each helix, positions toward the N-terminus receive lower numbers; positions toward the C-terminus receive higher numbers. This ligand-centric definition ensures that the anchors fall within the most structurally conserved region of each helix---the retinal-binding pocket---providing a stable reference even across distantly related families.
\autoref{fig:mogrn-1c3w} shows the system on bacteriorhodopsin (PDB: 1C3W\cite{luecke1999}). The Schiff base lysine K216 sits at 7.50, the counterions D85 and D212 at 3.45 and 7.46, and the TM3 motif residues that determine ion pump specificity at their respective positions. These residues, traditionally referenced by protein-specific numbers that differ in every homolog, become directly comparable across the superfamily.

\begin{figure}[H]
  \centering
  \includegraphics[width=0.55\textwidth]{\dir/fig_mogrn_1c3w.jpg}
  \caption[Key MOGRN positions on bacteriorhodopsin]{%
    Bacteriorhodopsin (PDB: 1C3W) with key MOGRN positions highlighted. The Schiff base lysine K216 (7.50), counterions D85 (3.45) and D212 (7.46), and TM3 residues T89 (3.49), W86 (3.46), and D96 (3.56) are shown as sticks. Retinal in rust.}
  \label{fig:mogrn-1c3w}
\end{figure}

The TM3 motif at positions 3.45--3.49--3.56 illustrates what standardized coordinates reveal (\autoref{tab:tm3-motif}). 

\begin{table}[tb]
  \caption[TM3 motif across microbial rhodopsin functional classes]{TM3 motif at positions 3.45--3.49--3.56 across microbial rhodopsin functional classes.}
  \label{tab:tm3-motif}
  \centering
  \begin{tabular}{lllll}
    \toprule
    Function & 3.45 & 3.49 & 3.56 & Example \\
    \midrule
    Outward H$^+$ pump & D & T & D & HsBR \\
    Inward H$^+$ pump (SzR) & F & S & E & SzR4 \\
    Inward H$^+$ pump (XeR) & D & T & A/L/S & NsXeR \\
    Cl$^-$ pump (archaeal) & T & S & A/D & NpHR \\
    Cl$^-$ pump (bacterial) & N & T & Q & NmClR \\
    Na$^+$ pump & N & D & Q & KR2 \\
    \bottomrule
  \end{tabular}
\end{table}

These three positions encode functional identity across ion-pumping rhodopsins. Outward proton pumps carry DTD (aspartate--threonine--aspartate), inward proton pumps carry FSE or DTA, chloride pumps carry TSA or NTQ, and sodium pumps carry NDQ. Before MOGRN, identifying these motifs meant structural alignment for each new protein. In standardized coordinates, they become queryable signatures. For instance, in the MOGRN system, additional functional positions include spectral tuning switches at 3.53, 4.51, 6.54, and~7.49, the retinal planarity determinant at 4.54 (the non-G rule), and lateral fenestration sites at 5.44 and~5.47.

The reference table was built from 129 structures---69 experimental and 60 computationally predicted---spanning all major functional classes. To annotate a new sequence, the GRN Processor finds the highest-identity reference in this table, aligns the query against it, and transfers GRN positions from the reference. Predicted structures achieved sub-angstrom accuracy for the binding pocket (\SIrange{0.5}{0.8}{\angstrom} iRMSD), confirming that current structure prediction methods are reliable for MOGRN annotation. The system accommodates non-canonical architectures: heliorhodopsins with inverted membrane topology, enzyme-fused rhodopsins with an additional TM0 helix, and local structural distortions that produce gaps or insertions. 

Separately, approximately 40,000 non-redundant sequences were aggregated from genomic and metagenomic sources---independently of the Type~II opsin dataset used in LAMBDA---and annotated with MOGRN positions, identifying 31 sequence clusters: 14 containing characterized rhodopsins and 17 containing only uncharacterized sequences, representing unexplored functional diversity.

The curated reference table enables GRN annotation for both opsin families: the GRN Processor annotates Type~I opsins using the MOGRN table, while Type~II opsins use Ballesteros--Weinstein positions. Both families now map to helix-relative coordinates anchored to the retinal pocket, so binding pocket graphs can be built for any opsin regardless of family.

\chapter{MOGRN --- A Generic Residue Numbering System for Microbial Rhodopsins}
\label{ch:mogrn}

\chapterabstract{The research contribution summarized here is presented in the accompanying publication: ``A Generic Residue-Numbering System for Microbial Rhodopsins --- Unifying Structural Frameworks and Functional Mapping.'' I am a co-first author on this work. The full manuscript is appended to this thesis.}


\section{Introduction}
\label{sec:mogrn-introduction}

Position~85 in bacteriorhodopsin is the counterion---the negatively charged aspartate that stabilizes the protonated Schiff base. In channelrhodopsin-2, the equivalent function sits at position~123. The numbers differ because each lineage accumulated different insertions and deletions over billions of years. Recognizing that these two residues serve the same role requires structural comparison; sequence numbering alone cannot reveal it.

GPCRs solved this problem decades ago. Ballesteros--Weinstein numbering assigns each position a helix-relative identifier---helix number and offset from the most conserved residue---creating coordinates that translate across the entire superfamily. GPCRdb built on this foundation\cite{kooistra2021}. When a researcher reports that a mutation at position~3.32 affects ligand binding, every GPCR researcher knows where that is. Microbial rhodopsins---functionally and taxonomically diverse---had nothing equivalent. Laboratories developed ad~hoc conventions, referencing bacteriorhodopsin numbering, channelrhodopsin-2, or the C1C2 chimera. Identity between major classes falls as low as 10--12\%\cite{degrip2022}, too low for reliable sequence alignment. A numbering system for microbial rhodopsins had to be anchored to structure.

We developed MOGRN (Microbial Opsin Generic Residue Numbering), anchored to the one feature all microbial rhodopsins share: the retinal-binding pocket. The system was validated on 129 structures spanning all major functional classes and applied to approximately 40,000 sequences. The key elements follow; the full methodology, results, and discussion are in the appended publication.


\section{Summary}
\label{sec:mogrn-summary}

MOGRN follows the Ballesteros--Weinstein convention: each position receives a two-part identifier indicating the helix (1--7) and the position relative to a helix-specific anchor. The anchor on each helix is the residue closest to retinal, designated X.50. The Schiff base-forming lysine on helix~7 becomes position~7.50 by definition. Positions toward the N-terminus receive lower numbers; positions toward the C-terminus receive higher numbers. This ligand-centric definition ensures that the anchors fall within the most conserved region of each helix---the retinal-binding pocket---providing a stable reference even across distantly related families.

\autoref{fig:mogrn-1c3w} shows the system on bacteriorhodopsin (PDB: 1C3W). The Schiff base lysine K216 sits at 7.50, the counterions D85 and D212 at 3.45 and 7.46, and the TM3 motif residues that determine ion pump specificity at their respective positions. These residues, traditionally referenced by protein-specific numbers that differ in every homolog, become directly comparable across the superfamily.

\begin{figure}[tb]
  \centering
  % \includegraphics[width=0.8\textwidth]{\dir/figure_3_1.pdf}
  \caption[Key MOGRN positions on bacteriorhodopsin]{%
    Bacteriorhodopsin (PDB: 1C3W) with key MOGRN positions highlighted. The Schiff base lysine K216 (7.50), counterions D85 (3.45) and D212 (7.46), and TM3 motif residues T89 (3.49) and D96 (3.56) are shown as sticks. Retinal in rust.}
  \label{fig:mogrn-1c3w}
\end{figure}

The TM3 motif at positions 3.45--3.49--3.56 illustrates what standardized coordinates reveal (\autoref{tab:tm3-motif}). These three positions encode functional identity across ion-pumping rhodopsins. Outward proton pumps carry DTD (aspartate--threonine--aspartate), inward proton pumps carry FSE or DTA, chloride pumps carry TSA or NTQ, and sodium pumps carry NDQ. Before MOGRN, identifying these motifs meant structural alignment for each new protein. In standardized coordinates, they become queryable signatures. \worktodo{Verbose: the grab-bag of functional positions (3.53, 4.51, 6.54, 7.49, 4.54, 5.44, 5.47) is listed without explanation of why each matters. Consider trimming or adding brief justification.}Additional functional positions include spectral tuning switches at 3.53, 4.51, 6.54, and~7.49, the retinal planarity determinant at 4.54 (the non-G rule), and lateral fenestration sites at 5.44 and~5.47.

\begin{table}[tb]
  \caption[TM3 motif across microbial rhodopsin functional classes]{TM3 motif at positions 3.45--3.49--3.56 across microbial rhodopsin functional classes.}
  \label{tab:tm3-motif}
  \centering
  \begin{tabular}{lllll}
    \toprule
    Function & 3.45 & 3.49 & 3.56 & Example \\
    \midrule
    Outward H$^+$ pump & D & T & D & HsBR \\
    Inward H$^+$ pump (SzR) & F & S & E & SzR4 \\
    Inward H$^+$ pump (XeR) & D & T & A/L/S & NsXeR \\
    Cl$^-$ pump (archaeal) & T & S & A/D & NpHR \\
    Cl$^-$ pump (bacterial) & N & T & Q & NmClR \\
    Na$^+$ pump & N & D & Q & KR2 \\
    \bottomrule
  \end{tabular}
\end{table}

The system was validated against 129 structures---69 experimental and 60 computationally predicted---spanning all major functional classes. Predicted structures achieved sub-angstrom accuracy for the binding pocket (\SIrange{0.51}{0.75}{\angstrom} iRMSD), confirming that current structure prediction methods are reliable for MOGRN annotation. The system accommodates non-canonical architectures: heliorhodopsins with inverted membrane topology, enzyme-fused rhodopsins with an additional TM0 helix, and local structural distortions that produce gaps or insertions. Application to approximately 40,000 non-redundant sequences from genomic and metagenomic sources identified 31 sequence clusters---14 containing characterized rhodopsins and 17 containing only uncharacterized sequences, representing unexplored functional diversity.

\worktodo{Verbose: the following five sentences each restate ``MOGRN provides standardized coordinates'' from a different angle. Consider condensing to two sentences.}MOGRN gives microbial rhodopsins the standardized coordinates that GPCRs have had for decades. The structural alignment of 129 microbial rhodopsins produces a curated reference table that the GRN Processor uses to annotate type~I opsins---the same workflow that Ballesteros--Weinstein tables provide for type~II. Both families now map to helix-relative coordinates anchored to the retinal pocket. Binding pocket graphs---where each node carries a positional label rather than a protein-specific residue number---can be built for any opsin, regardless of family. The labels are transferable: position~3.45 in a proton pump, a channelrhodopsin, or any other microbial rhodopsin refers to the same structural location relative to retinal.
